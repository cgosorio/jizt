\capitulo{6}{Trabajos relacionados}

A continuación, enumeraremos algunos proyectos relacionados con nuestro trabajo.

\subsubsection{Bert Extractive Summarizer}

Este proyecto \emph{open-source} implementa un generador de resúmenes extractivos haciendo uso del modelo BERT \cite{devlin19} de Google para la codificación de palabras, y aplicando \emph{clústering} por \emph{k-means} para determinar las frases que se incluirán en el resumen. Este proceso se detalla en \cite{miller19}.

El generador de resúmenes puede ser \emph{dockerizado}, pudiéndose ejecutar como servicio, proporcionando una REST API para solicitar los resúmenes. El autor ofrece \emph{endpoints} gratuitos con limitaciones a la hora de realizar peticiones, y \emph{endpoints} privados de pago para aquellos particulares o empresas que requieran de mayores prestaciones.

Se puede acceder al proyecto a través del siguiente enlace: \\
\href{https://github.com/dmmiller612/bert-extractive-summarizer}{\texttt{{\small https://github.com/dmmiller612/bert-extractive-summarizer}}}.


\bigskip
\subsubsection{ExplainToMe}

ExplainToMe es un proyecto también \emph{open-source} centrado en la generación de resúmenes extractivos de páginas \emph{web}, permitiendo cómodamente pegar y copiar el \emph{link} de la \emph{web} que se quiere resumir.

Emplea el algoritmo de TextRank \cite{mihalce04}, el cual a su vez está inspirado en el conocido PageRank \cite{page99}, el algoritmo basado en grafos que empleaba originalmente Google en su motor de búsqueda. En su caso, TextRank aplica los principios del algoritmo de Google a la extracción de las frases más importantes de un texto.

Como en el caso anterior, también implementa una API REST.

El proyecto no ha sido actualizado desde finales de 2018. Se puede visitar a través de:
\href{https://github.com/jjangsangy/ExplainToMe/tree/master}{\texttt{{\small https://github.com/jjangsangy/ExplainToMe/tree/master}}}.


\bigskip
\subsubsection{SMMRY}

Se trata de uno de las primeras opciones que aparecen en los motores de búsqueda a la hora de buscar <<\emph{summarizers}>>. También genera resúmenes extractivos, aunque a diferencia de los anteriores, no es un proyecto \emph{open-source}.

Destacan su velocidad (\emph{cachea} los textos resumidos recientemente), y sus múltiples opciones de resumen, como por ejemplo: ignorar preguntas, exclamaciones o frases entrecomilladas en el texto original, o la generación de mapas de calor en función de la importancia de las frases incluidas en el resumen.

Sin embargo, los resúmenes están compuestos de frases literales ordenadas cronológicamente en función de su importancia, por lo que la cohesión entre las mismas puede ser frágil e incluso, con frecuencia, se habla de personas o entidades que no han sido introducidas previamente en el resumen, pudiendo dificultar la comprensión del mismo.

En su página \emph{web} no se explicita el algoritmo concreto que se emplea, pero prestando atención a la descripción del proceso proporcionada \cite{smmry}, parecen emplear igualmente PageRank.

Se puede acceder a SMMRY en:
\href{https://smmry.com/}{\texttt{{\small https://smmry.com/}}}.

\newpage
\subsubsection{Tabla comparativa}

\vspace{0.5cm}
\begin{table}[h]\label{tabla:comparativa}
	\centering
	\begin{tabular}{lcccc}
		\toprule
		\multirow{2}{*}{Caraterísticas} & \multirow{2}{*}{\textbf{\href{https://www.jizt.it/}{\small{JIZT}}}} & \scriptsize{\href{https://github.com/dmmiller612/bert-extractive-summarizer}{Bert Extracive}} & \multirow{2}{*}{\scriptsize{\href{https://github.com/jjangsangy/ExplainToMe}{ExplainToMe}}} & \multirow{2}{*}{\href{https://smmry.com/}{\small{SMMRY}}} \\
		& & \scriptsize{\href{https://github.com/dmmiller612/bert-extractive-summarizer}{Summarizer}} & & \\
		\midrule
		\small{Tipo de resumen$^1$} & {\small Abstractivo} & {\small Extractivo} & {\small Extractivo} & {\small Extractivo} \\
		\scriptsize{Tiempo resumen corto$^2$} & \small{\textasciitilde 20 seg.} & \small{\textasciitilde 6 seg.} & {\small \textasciitilde 9 seg.} & {\small \textasciitilde 3 seg.} \\
		\scriptsize{Tiempo resumen largo$^3$} & {\small \textasciitilde 4 min.} & \scriptsize{No disponible$^4$} & {\small Error} & {\small \textasciitilde 5 seg.} \\
		{\small Ajustes básicos} & \cellcolor{green!25} {$\checkmark$} & \cellcolor{green!25} {$\checkmark$} & \cellcolor{green!25} {$\checkmark$} & \cellcolor{green!25} {$\checkmark$} \\
		\small{Ajustes avanzados} & \cellcolor{green!25} {$\checkmark$} & \cellcolor{red!25} $\times$ & \cellcolor{red!25} $\times$ & \cellcolor{green!25} {$\checkmark$} \\
		\scriptsize{Entrada: texto plano} & \cellcolor{green!25} {$\checkmark$} &  \cellcolor{green!25} {$\checkmark$} &  \cellcolor{red!25} $\times$ & \cellcolor{green!25} {$\checkmark$} \\
		\small{Entrada: URL} & \cellcolor{yellow!25} {\small \hspace{-0.3cm} Próximamente} &  \cellcolor{red!25} $\times$ &  \cellcolor{green!25} {$\checkmark$} &  \cellcolor{green!25} {$\checkmark$} \\
		\small{Entrada: fichero} & \cellcolor{yellow!25} {\small \hspace{-0.3cm} Próximamente} &  \cellcolor{red!25} $\times$ & \cellcolor{red!25} $\times$ &  \cellcolor{green!25} {$\checkmark$} \\
			\small{Entrada: imagen} & \cellcolor{yellow!25} {\small \hspace{-0.3cm} Próximamente} &  \cellcolor{red!25} $\times$ & \cellcolor{red!25} $\times$ &  \cellcolor{red!25} $\times$ \\
		\scriptsize{Soporte multi-modelo$^5$} & \cellcolor{yellow!25} {\small \hspace{-0.3cm} Próximamente} & \cellcolor{red!25} $\times$ & \cellcolor{red!25} $\times$ & \cellcolor{red!25} $\times$ \\
		\scriptsize{Soporte multi-tarea$^6$} & \cellcolor{yellow!25} {\small \hspace{-0.3cm} Próximamente} & \cellcolor{red!25} $\times$ & \cellcolor{red!25} $\times$ & \cellcolor{red!25} $\times$ \\
		\small{API REST} & \cellcolor{green!25} {$\checkmark$} & \cellcolor{green!25} {$\checkmark$} & \cellcolor{green!25} {$\checkmark$} & \cellcolor{green!25} {$\checkmark$} \\
		\small{Arquitectura} & {\small Microservicios} & {\small Monolítica} & {\small Monolítica} & ? \\
		\small{Plataforma} & \scriptsize{Multiplataforma$^7$} & {\small Web} & {\small Web} & {\small Web} \\
		\emph{Open-source} & \cellcolor{green!25} {$\checkmark$}& \cellcolor{green!25} {$\checkmark$} & \cellcolor{green!25} {$\checkmark$} & \cellcolor{red!25} $\times$ \\
		\emph{Gratuito} & \cellcolor{green!25} {$\checkmark$}& \cellcolor{yellow!25} \small{Limitado} & \cellcolor{green!25} {$\checkmark$} & \cellcolor{yellow!25} \small{Limitado} \\
		\small{Proyecto activo} & \cellcolor{green!25} {$\checkmark$}& \cellcolor{green!25} {$\checkmark$} & \cellcolor{red!25} $\times$ & \cellcolor{green!25} {$\checkmark$} \\
		\bottomrule
		\multicolumn{5}{l}{\scriptsize{1. En los resúmenes \emph{abstractivos}, se toman las frases literales del texto original. En los \emph{ex-}}} \\
		\multicolumn{5}{l}{\hspace{0.26cm} \scriptsize{\emph{tractivos}, se añaden palabras o expresiones nuevas.}} \\
		\multicolumn{5}{l}{\scriptsize{2. Texto de entrada con \textasciitilde 6.500 caracteres.}} \\
		\multicolumn{5}{l}{\scriptsize{3. Texto de entrada con \textasciitilde 90.000 caracteres.}} \\
		\multicolumn{5}{l}{\scriptsize{4. La versión gratuita está limitada. No hemos tenido acceso a la versión completa.}} \\
		\multicolumn{5}{l}{\scriptsize{5. Capacidad de generar resúmenes utilizando diferentes modelos.}} \\
		\multicolumn{5}{l}{\scriptsize{6. Capacidad de realizar otras tareas de NLP diferentes a la generación de resúmenes.}} \\
		\multicolumn{5}{l}{\scriptsize{7. Soporte nativo para Android, iOS y \emph{web}. Pronto, soporte para Linux, macOS y Windows.}} \\
		\bottomrule
	\end{tabular}	
	\caption{Comparativa de las características ofrecidas por las diferentes alternativas para la generación de resúmenes.}
	\label{table:comparativa}
\end{table}


